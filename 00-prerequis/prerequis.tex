\documentclass{beamer}
\usepackage{ragged2e}
\usepackage[frenchb]{babel}
\usetheme[subsectionpage=progressbar]{metropolis}
\usepackage{fancyvrb}
\usepackage{graphicx}
\usepackage{listings}
\usepackage{xcolor}

\setcounter{tocdepth}{1}
\apptocmd{\frame}{}{\justifying}{} % Allow optional arguments after frame.

\colorlet{punct}{red!60!black}
\definecolor{background}{HTML}{EEEEEE}
\definecolor{delim}{RGB}{20,105,176}
\colorlet{numb}{magenta!60!black}

\lstdefinelanguage{json}{
  basicstyle=\normalfont\ttfamily,
  numbers=left,
  numberstyle=\scriptsize,
  stepnumber=1,
  numbersep=8pt,
  showstringspaces=false,
  breaklines=true,
  frame=lines,
  backgroundcolor=\color{background},
  literate=
  *{0}{{{\color{numb}0}}}{1}
  {1}{{{\color{numb}1}}}{1}
  {2}{{{\color{numb}2}}}{1}
  {3}{{{\color{numb}3}}}{1}
  {4}{{{\color{numb}4}}}{1}
  {5}{{{\color{numb}5}}}{1}
  {6}{{{\color{numb}6}}}{1}
  {7}{{{\color{numb}7}}}{1}
  {8}{{{\color{numb}8}}}{1}
  {9}{{{\color{numb}9}}}{1}
  {:}{{{\color{punct}{:}}}}{1}
  {,}{{{\color{punct}{,}}}}{1}
  {\{}{{{\color{delim}{\{}}}}{1}
  {\}}{{{\color{delim}{\}}}}}{1}
  {[}{{{\color{delim}{[}}}}{1}
  {]}{{{\color{delim}{]}}}}{1},
}

\title{Prérequis du cours de programmation web}
\date{}
\author{Lionel \textsc{Kitihoun}}

\begin{document}
\begin{frame}[plain]
\maketitle
\end{frame}

\begin{frame}{Sommaire}
\tableofcontents
\end{frame}

\section{Algorithmique}

\begin{frame}{Algorithmique}
Un bon niveau en algorithmique est requis.

Avoir un bon niveau en algorithmique permet de pouvoir apprendre n'importe langage avec une facilité relative.
\end{frame}

\section{HTML, CSS et JS}
\begin{frame}{HTML, CSS et JS}
\begin{itemize}
  \item HTML 5
  \item CSS
  \item Bibliothèques CSS (optionnel, mais utile)
  \begin{itemize}
    \item Bootstrap
    \item Bulma
    \item Tailwind
  \end{itemize}
  \item JavaScript (optionnel)
\end{itemize}
\end{frame}

\section{PHP}
\begin{frame}{PHP}
\begin{center}
  \begin{itemize}
    \item Comprendre le fonctionnement de PHP en tant que langage de script CGI.
    \item Savoir générer du contenu HTML avec PHP.
    \item Savoir traiter les données de formulaires.
    \item Connaissance de l'API PDO.
    \item POO (optionnel).
  \end{itemize}
\end{center}
\end{frame}

\section{SQL}
\begin{frame}{SQL}
\begin{center}
\begin{itemize}
  \item DDL.
  \item DML.
  \item Pouvoir créer une base de données sur un serveur MySQL ou PostgreSQL et y exécuter des requêtes.
\end{itemize}
\end{center}
\end{frame}

\section{Packages XAMP}
\begin{frame}{Packages XAMP}
\begin{itemize}
  \item Savoir installer et utiliser un package XAMP.
    \begin{itemize}
      \item Wamp Server.
      \item EasyPHP.
      \item XAMPP.
      \item Bitnami.
      \item Laragon.
    \end{itemize}
  \item Connaître les outils PHPMyAdmin, HeidiSQL, PgAdmin, etc.
\end{itemize}
\end{frame}

\section{Autres éléments importants}
\begin{frame}{Autres éléments importants}
\begin{itemize}
  \item Savoir recopier un code sans erreur.
  \item Savoir lire les messages d'erreurs PHP pour identifier les erreurs et les corriger.
\end{itemize}
\end{frame}
\end{document}
